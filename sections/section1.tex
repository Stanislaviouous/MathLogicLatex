\documentclass[../main.tex]{subfiles}
\usepackage[utf8]{inputenc}
\usepackage[russian]{babel}
\usepackage{setspace, amsmath}
\usepackage{array}
\usepackage{amssymb}
\graphicspath{{\subfix{../images/}}}

\newtheorem{defin}{Определение:}
 

\begin{document}

% 1
\section{Понятие высказывания}
\textbf{Определение:
} Высказывание - это простое утдвердительное предложение (содержит одну мысль), о котором   можно сказать истинно они или ложно.
\textbf{Пример:
} Люди сметрны!

%2
\section{Логическое значение высказывания}
\textbf{Факт:
} Существуют два значения для высказываний — «истина» (1) и «ложь» (0).
\begin{equation*}
    \lambda (p) = 
    \begin{cases}
         0,$ если $p -$ ложно$ \\
         1,$ если $p -$ истинно$  
    \end{cases}
\end{equation*}

%3 Убрать пропозициональные переменные
\section{Классификация высказываний}
\textbf{Определение:
} Тавтология - высказывание принимающее логическое значение 1 при любом наборе значений пропозициональных переменных. \\
\textbf{Определение:
} Тождественная ложь - высказывание принимающее логическое значение 0 при любом наборе значений пропозициональных переменных. \\ 
\textbf{Определение:
} Выполнимое - высказывание принимающее логическое значение 1 хотя бы на одном наборе значений пропозициональных переменных. \\
\textbf{Определение:
} Опровержимое - высказывание принимающее логическое значение 0 хотя бы на одном наборе значений пропозициональных переменных. \\

%4
\section{Понятие логической функции}
\textbf{Определение:
} Логическая функция — это отображение множества высказываний в множество логических значений.\\
Логические операции позволяют построить сложные высказывания из данных высказываний, при котором истинностное значение сложного высказывания полностью определяется значениями исходных высказываний. \\
\textbf{Частный случай:
} Логические связки - булевые функции.

%5
\section{Таблицы истинности логические функции}
\textbf{Определение:
} Таблица истинности — это таблица, которая показывает значения логической функции в зависимости от значений ее аргументов (какое булево значение будет возвращено функцией при различных комбинациях значений ее аргументов). \\
\textbf{Пример:
} 
\begin{center}
    \begin{tabular}{ | m{3em} | m{3em}| m{3em} | } 
        \hline
        A& B & A $\circ$ B \\ 
        \hline
        0 & 0 & 1 \\ 
        \hline
        0 & 1 & 1 \\ 
        \hline
        1 & 0 & 0 \\ 
        \hline
        1 & 1 & 0 \\
        \hline 
    \end{tabular}
\end{center}

%6
\section{Основные логические функции}
\begin{center}
    \begin{tabular}{| m{0.5em} | m{5em} | m{5em}| m{8em} | m{0.5em} | m{0.5em}| m{0.5em} | m{0.5em} | } 
        \hline
        A B & Функция & Пример & Название & 0 0 & 0 1 & 1 0 & 1 1\\ 
        \hline
        1 & $\lnot$ & $\lnot A$ & Инверсия & 1 &   & 0 &   \\ 
        \hline
        2 & $\lor$ &  $A \lor B$ & Конъюнкция & 0 & 1 & 1 & 1\\ 
        \hline
        3 & $\land$ & $A \land B$ & Дизъюнкция & 0 & 0 & 0 & 1\\ 
        \hline
        4 & $\rightarrow$ & $A \rightarrow B$ & Импликация & 1 & 1 & 0 & 1\\
        \hline
        5 & $\leftrightarrow$ & $A \leftrightarrow B$& Эквиваленция & 1 & 0 & 0 & 1\\
        \hline
        6 & $\oplus $ & $A \oplus B$& по модулю 2 & 0 & 1 & 1 & 0\\
        \hline
        7 & $ / $ & $A / B$& Шеффера штрих & 1 & 1 & 1 & 0\\
        \hline
        8 & $ \downarrow $ & $A \downarrow  B$& Стрелка Пирса & 1 & 0 & 0 & 0\\
        \hline
    \end{tabular}
\end{center}

%7
\section{Законы алгебры высказываний}
\textbf{Определение:
}Алгебра высказываний: $A = \{P, \varPhi \}$, где $P = \{$ $\lnot$, $\lor$, $\land$, $\rightarrow$, $\leftrightarrow$ $\}$ - булевые связки, а $\varPhi -$ алфавит высказываний (пропозициональных переменных.\\
Законы:
\begin{center}
    \begin{tabular}{| m{10em} | m{10em} | m{10em}|} 
        \hline
        Идемпотентность & A $\lor$ A = A & A $\land$ A = A\\ 
        \hline
        Коммутативность & A $\lor$ B = B $\lor$ A &  A $\land$ B = B $\land$ A\\ 
        \hline
		Ассоциативность & (A $\lor$ B) $\lor$ C = A $\lor$ (B $\lor$ C) & (A $\land$ B) $\land$ C = A $\land$ (B $\land$ C) \\ 
        \hline
        Дистрибутивность &  A $\land$ (B $\lor$ C) = (A $\land$ B) $\lor$ (A $\land$ C) & A $\lor$ (B $\land$ C) = (A $\lor$ B) $\land$ (A $\lor$ C)\\ 
        \hline
        Действия с константами & A $\lor$ 1 = 1  A $\lor$ 0 = A & A $\land$ 1 = A  A $\lor$ 0 = 0\\
        \hline
		Исключенного третьего & $\lnot$ A $\lor$ A = 1 & \\
        \hline
        Двойного отрицания & $\lnot$($\lnot$A) = A & \\
        \hline
        Противоречия & & $\lnot$A $\land$ A = 0\\
        \hline
		Де Моргана & $\lnot$(A $\lor$ B) = $\lnot$A $\lor$ $\lnot$ B & $\lnot$ (A $\land$ B) = $\lnot$A $\lor$ $\lnot$B\\
        \hline
    \end{tabular}
\end{center}


%8
\section{Понятие силлогизма}
\textbf{Определение:
}Тавтологии представляют собой схемы построения истинных высказываний, независимо от содержания и истинности составляющих высказываний. Основное значение тавтологий состоит в том, что некоторые из них предоставляют правильные способы построения умозаключений, т.е. такие способы, которые от истинных посылок всегда приводят к истинным выводам. \\
\textbf{Определение:
} Силлогизм — это цепочка (форма логического вывода), основанная на трех законах: законе тождества, законе противоречия и законе исключенного третьего. \[(A \rightarrow B) \land (B \rightarrow C) \to {(A \rightarrow C)}\]\\
Задействует логическое следование: $\stackrel{assumptions}{(A \rightarrow B), (B \rightarrow C)} \vdash \stackrel{consequence}{(A \rightarrow C)}$, 
$\small assumptions$ - посылки, $\small consequence$ - следствие. \\
\textbf{Пример:
}``Все люди смертны, следовательно, Сократ смертен.'' \\
Здесь ``Все люди смертны'' - посылка и ``Сократ - человек'' - посылка, а ``Сократ смертен'' — следствие.\\
% \[
%     F_1, F_2, F_3 \cdots F_n \models G, \text {только если } \Big( \lambda (F_1) = \lambda (F_2) = \cdots = 1 \Rightarrow \lambda (G) \Big)
% \]
\textbf{
    Определение:
} Энтимема - силлогизм с пропущенной посылкой или заключением. \\
\textbf{
	Определение:
} Эпихейрема – сложносокращенный силлогизм обе посылки которого – сокращенные силлогизмы (энтимемы). \\
соединение сокращённых силлогизмов, в которых опущена или большая, или меньшая посылка. \\
\textbf{
	Определение:
} Полисиллогизм - два или более простых силлогизма, которые связаны между собой так, что вывод одного из них служит посылкой другого. \\
\textbf{
	Пример:
}(Все продукты, содержащие витамины, полезны. Все фрукты содержат витамины.) \\
(Все фрукты полезны. Яблоко - это фрукт.)
Яблоки полезны. \\
\textbf{
	Определение:
} Сорит - это сложносокращенный силлогизм; полисиллогизм с пропущенной посылкой последующего силлогизма, являющейся
выводом предыдущего силлогизма. \\
\textbf{
	Пример:
} (Все продукты, содержащие витамины, полезны. Все фрукты содержат витамины.) \\
($\textit{(пропущен вывод, что все фрукты полезны)}$ Яблоко - это фрукт.)
Яблоки полезны.

%9
\section{Понятие контрапозиции}
\textbf{Определение:
}Тавтологии представляют собой схемы построения истинных высказываний, независимо от содержания и истинности составляющих высказываний. Основное значение тавтологий состоит в том, что некоторые из них предоставляют правильные способы построения умозаключений, т.е. такие способы, которые от истинных посылок всегда приводят к истинным выводам. \\
\textbf{
	Определение:
}Контрапозиция — это закон, логический инструмент, который заключается в отрицании какого-либо утверждения или явления. Позволяющих с помощью отрицания менять местами посылку и следствие высказывания.
\[A \to B \vdash \lnot B \to \lnot A\]
\textbf{
	Пример:
}``Без огня нет дыма'', значит ``Если есть дым - есть огонь''

%10
\section{Понятие формулы логики высказываний ЛВ}
\textbf{Определение:
}Алгебра высказываний: $A = \{P, \varPhi \}$, где $P = \{$ $\lnot$, $\lor$, $\land$, $\rightarrow$, $\leftrightarrow$ $\}$ - булевые связки, а $\varPhi -$ алфавит высказываний (пропозициональных переменных.\\
\textbf{
	Определение:
}Формула такой логики(алгебры) - высказывание, составленное из элементов алфавита: пропозициональных переменных, логических связок и скобок.\\
Справедливо:
\begin{itemize}
	\item Всякая пропозициональная переменная есть формула.
	\item Если $p$ и $q$ - формулы, то выражения $(\lnot p), (p \lor q), (p \land q), (p \to q), (p \leftrightarrow q)$ - тоже формулы
	\item Других формул не существует
\end{itemize}
\textbf{Пример:
}$(A \rightarrow B) \to (B \rightarrow C)$

%11
\section{Существенные и фиктивные переменные формулы ЛВ}
\textbf{Определение:
} Функция $F(X_1, X_2 … X_{i-1}, X_i, X_{i+1} \ldots X_n)$ существенно зависит от переменной $X_i$, если у переменных $X_1 \ldots, X_{i-1}, X_{i+1} \ldots X_n$ существует такой набор значений $\alpha_1, \alpha_2 \ldots \alpha_{i-1}, \alpha_{i+1} \ldots \alpha_n$, что выполняется неравенство:  
\[F(\alpha_1, \alpha_2 \ldots \alpha_{i-1}, 0, \alpha_{i+1} \ldots \alpha_n) \ne F(\alpha_1,\alpha_2, \ldots \alpha_{i-1}, 1, \alpha_{i+1} \ldots \alpha_n)\]
\\
\textbf{Определение:
} Переменную $X_i$ называют фиктивной, если для любого набора значений $\alpha_1, \alpha_2 \ldots \alpha_{i-1}, \alpha_{i+1} \ldots \alpha_n$ переменных $X_1, X_2 \ldots X_{i-1}, X_{i+1} \ldots X_n$ выполняется равенство:
\[f(\alpha_1, \alpha_2 \ldots \alpha_{i-1}, 0, \alpha_{i+1} \ldots \alpha_n) = f(\alpha_1,\alpha_2, \ldots \alpha_{i-1}, 1, \alpha_{i+1} \ldots \alpha_n)\]
Существенные переменные влияет на значение формулы ЛВ, фиктивная - нет.

%12
\section{Теорема о признаке равносильности формул ЛВ}
\textbf{Определение:
} Тавтологии представляют собой схемы построения истинных высказываний, независимо от содержания и истинности составляющих высказываний.\\
\textbf{
	Определение:
} Формулы $F(X_1,X_2,\ldots,X_n)$ и $H(X_1,X_2,\ldots,X_n)$ алгебры высказываний равносильны(эквивалентны), если при любых значениях пропозициональных переменных логические значения формул F и H совпадают. Для указания равносильности формул используют $F\cong H$. Равносильность формул для любых конкретных высказываний $A_1,A_2,\ldots,A_n$:\\
\[F\cong H\quad\leftrightarrow\quad \lambda \bigl(F(A_1, A_2,\ldots,A_n)\bigr)= \lambda\bigl(H(A_1,A_2,\ldots,A_n)\bigr)\]
\textbf{
	Теорема:
} O признаке равносильности формул: Две формулы F и H алгебры высказываний равносильны тогда и только тогда, когда формула $F\leftrightarrow H$ является тавтологией:\\
\[F\cong H\quad \Leftrightarrow\quad \vDash F\leftrightarrow H\]
\textbf{
	Доказательство:
} Если $F\cong H$, то по определению $\lambda\bigl(F(A_1,\ldots,A_n)\bigr)= \lambda\bigl(H(A_1,\ldots,A_n)\bigr)$ для любых высказываний $A_1,\ldots,A_n$. Тогда $\lambda\bigl(F(A_1,\ldots,A_n)\bigr)\leftrightarrow \lambda\bigl(H(A_1,\ldots,A_n)\bigr)=1$, откуда на основании $\lambda(P\leftrightarrow Q)= \lambda(P) \leftrightarrow \lambda(Q)$, $\lambda\bigl(F(A_1, \ldots,A_n) \bigr)\leftrightarrow \lambda\bigl(H(A_1,\ldots,A_n)\bigr)=1$ для любых $A_1,\ldots,A_n$. Последнее означает по определению тавтологии, что $\vDash F\leftrightarrow H$. Обратными рассуждениями доказывается утверждение: если $\vDash F\leftrightarrow H$, то $F\cong H$.

%13
\section{Понятие логического следования}
\textbf{Определение:
} Формула G - логическое следствие формул $F_1, F_2 \ldots F_m$, если она превращается в истинное высказывание при всякой такой подстановке вместо всех ее пропозициональных переменных ${X_1 \ldots X_n}$ конкретных высказываний, при которой в истинное высказывание превращаются все формулы $F_1(X_1 \ldots X_n),\ldots, F_m(X_1 \ldots X_n)$. Логическое следствие обозначается так: $F_1,\ldots,F_m\vDash G$. Формулы $F_1 \ldots F_m$ называются посылками для логического следствия G.
\[
    F_1, F_2, F_3 \cdots F_n \models G, \text {только если } \Big( \lambda (F_1) = \lambda (F_2) = \cdots = 1 \rightarrow \lambda (G) = 1 \Big)
\]
\textbf{
	Пример:
} Если $\stackrel{F_1}{\text{я предварительно получу 4}}$ и $\stackrel{F_2}{\text{выучу билеты}}$, \\то $\stackrel{G}{\text{буду сдавать экзамен по МЛТА}}$

%14
\section{Теорема о признаке логического следования}
\textbf{Теорема:
} о признаке логического следствия: Формула G будет логическим следствием формулы F тогда и только тогда, когда формула $F\to H$ является тавтологией: 
\[F\vDash H\Leftrightarrow\ \quad \vDash F\to H\]
\textbf{
	Доказательство:
} Необходимость. Дано: $F(X_1,\ldots,X_n)\vDash H(X_1,\ldots,X_n)$, т.е. если для набора высказываний $A_1,\ldots,A_n$ имеет место $\lambda\bigl(F(A_1,\ldots,A_n)\bigr)=1$, то $\lambda\bigl(H(A_1,\ldots,A_n)\bigr)=1$. Тогда для любого набора высказываний $A_1,\ldots,A_n$ имеет место равенство:
\[\lambda\bigl(F(A_1,\ldots,A_n)\bigr)\to \lambda \bigl(H(A_1,\ldots,A_n)\bigr)\]
Так как равенство не может быть равно нулю и $\lambda\bigl(F(A_1,\ldots,A_n)\bigr)\to \lambda\bigl(H(A_1,\ldots,A_n)\bigr) = \lambda\bigl(F(A_1,\ldots,A_n)\to H(A_1,\ldots,A_n)\bigr)=1$, то $\lambda\bigl(F(A_1,\ldots,A_n)\to H(A_1,\ldots,A_n)\bigr)=1$ для любых высказываний $A_1,\ldots,A_n$. Это означает, что формула $F(X_1,\ldots,X_n)\to H(X_1,\ldots,X_n)$ — тавтология, т.е. $\vDash F\to H$. \\
\textbf{
Доказательство:
} Достаточность. Из необходимости:
\[\lambda\bigl(F(A_1,\ldots,A_n)\bigr)\to \lambda\bigl(H(A_1,\ldots,A_n)\bigr)=1\]
Предположим теперь, что $\lambda\bigl(F(A_1, \ldots, A_n)\bigr)=1$. Тогда: $1\to \lambda\bigl(H(A_1, \ldots, A_n)\bigr)=1$, откуда $\lambda\bigl(H(A_1,\ldots,A_n)\bigr)=1$, ибо в противном случае $1\to 0=1$ — противоречие. Это значит (по определению логического следствия), что $F\vDash H$.

%15
\section{Свойства логического следования}
\textbf{Факт:
}Свойства логического следования между формулами алгебры высказываний:
\begin{itemize}
	\item 1) $F_1, F_2, \ldots, F_m\vDash F_i$ для $i=1 \ldots m$ - отношение логического следования рефлексивно
	\item 2) Если $F_1, F_2 \ldots F_m\vDash G_i$ и $G_1,G_2 \ldots G_p\vDash H$, то $F_1, F_2 \ldots F_m\vDash H$ \\
\end{itemize}

\end{document}
