\documentclass[../main.tex]{subfiles}
\usepackage[utf8]{inputenc}
\usepackage[russian]{babel}
\usepackage{setspace, amsmath}
\usepackage{array}
\usepackage{amssymb}
\graphicspath{{\subfix{../images/}}}

\begin{document}

%16
\section{Формализованное исчисление высказываний (ФИВ)}
\textbf{Определение:
} Язык ФИВ: пропозициональные переменные, логические связки ($\lnot,\to$), (,) — технические знаки. \\
\textbf{
	Определение:
} Формулы ФИВ (как в алгебре высказываний):
\begin{itemize}
	\item каждая пропозициональная переменная есть формула
	\item если $F_1 и F_2$ - формулы, то выражения $\lnot F_1, (F_1\to F_2)$ тоже формулы 
	\item никаких других формул нет 
\end{itemize}
\textbf{
	Определение:
} Система аксиом ФИВ:
\begin{itemize}
	\item \textbf{(A1)}$\quad F\to (G\to F)$
	\item \textbf{(A2)}$\quad \bigl(F\to (G\to H)\bigr)\to \bigl((F\to G)\to (F\to H)\bigr)$
	\item \textbf{(A3)}$\quad (\lnot G\to\lnot F)\to \bigl((\lnot G\to F)\to G\bigr)$
\end{itemize}
\textbf{Определение:
} Правило вывода - правило отсечения(\textit{modus ponens}):\[\frac{F, F \to G}{G}\]
$F \land G \cong \ \lnot(F \to \lnot G)$\\
$F \lor G \cong \ \lnot F \to G$

%17
\section{Свойства ФИВ}
\textbf{Факт:
}Полнота, непротиворечивость, разрешимость.\\
\textbf{Теорема:
} О полноте: Формула тогда и только тогда доказуема в формализованном исчислении высказываний, когда она является тавтологией алгебры высказываний:\[\vdash F\Leftrightarrow\vDash F\]
\textbf{Теорема:
} О полноте: Всё, что истинно, то выводимо, и обратно: 
\[\vDash F \ \Leftrightarrow \ \vdash F\]
\textbf{Теорема:
} О непротиворечивости: $F$ и $\lnot F$ - не могут одновременно быть теоремами данной аксиоматическиой теории, $\bigl(\vdash F \ \Leftrightarrow \ \nvdash (\lnot F)\bigr)$ -  $F$ тогда выводима, когд $\lnot F$ не выводима. \\
\textbf{Теорема:
} О разрешимости: ФИВ - разрешимая аксиоматическая теория - существует алгоритм, позволяющий для любого утверждения, сформулированного в терминах теории, ответить на вопрос, будет или нет это утверждение теоремой данной теории. \\

%18
\section{Понятие вывода (доказательства) формулы ФИВ}
\textbf{Определение:
}Вывод (доказательство) $\mathbf{F}$ из множества формул $\mathbf{\Gamma}$(множества гипотез, посылок вывода) - это такая конечная последовательность $B_1,B_2\ldots B_s$ формул, что каждая её формула - либо аксиома, либо формула из $\mathbf{\Gamma}$, либо получена из двух предыдущих формул этой последовательности по правилу MP, при этом последняя формула = $\mathbf{F}$. \\
Если есть вывод формулы $\mathbf{F}$ из множества $\mathbf{\Gamma}$,то $\mathbf{F}$ выводима из $\mathbf{\Gamma}$: $\Gamma\vdash$ $F$. \\
Если есть вывод формулы $\mathbf{F}$ из пустого множества гипотез, что $\mathbf{F}$ выводима из аксиом $\vdash F$ - теорема. \\
\textbf{Пример:
} $\vdash \ F \to F$
\begin{align*}
	\quad&\mathsf{(1)}\colon\quad F\to ((F\to F)\to F)															&\text{(A1)}\\
	\quad&\mathsf{(2)}\colon\quad \bigl(F\to ((F\to F)\to F)\bigr)\to \bigl((F\to (F\to F)) \to (F\to F)\bigr) 	&\text{(A2)}\\
	\quad&\mathsf{(4)}\colon\quad \bigl(F\to (F\to F)\bigr)\to (F\to F) 										&\text{(MP)}\\
	\quad&\mathsf{(4)}\colon\quad F\to (F\to F) 																&\text{(A1)}\\
	\quad&\mathsf{(5)}\colon\quad F\to F. 																		&\text{(MP)}
\end{align*}

%19
\section{Теорема о дедукции}
\textbf{Теорема:
} О дедукции: Если $F_1\ldots F_{m-1},F_m\vdash G$, то $F_1\ldots F_{m-1}\vdash F_m\to G$. В частности, если $F\vdash G, то \vdash F\to G$. \\
% \textbf{Пример:
% }
% \textbf{Доказательство:
% } Пусть последовательность формул
% $B_1, B_2 \ldots B_i \ldots B_s(1)$ - является выводом формулы G из гипотез $F_1,\ldots F_{m-1}, F_m$ (такая последовательность существует, поскольку, по условию, $F_1 \ldots,F_{m-1} F_m\vdash G$. Следовательно, формула $B_s$ есть формула G, т.е. $B_s\equiv G$ ($\equiv$ — знак графического совпадения, одинаковости формул). Рассмотрим последовательность формул:
% \[F_m\to B_1, F_m\to B_2, \ldots, F_m\to B_i, \ldots, F_m\to B_s\,.(2)\]
% На последнем месте данной последовательности стоит формула $F_m\to G$ (так как $B_s\equiv G$). Но эта последовательность, вообще говоря, не является выводом из гипотез $F_1\ldots F_{m-1}$. Тем не менее ее можно превратить в такой вывод, если перед каждой формулой последовательности добавить подходящие формулы. Для этого покажем методом математической индукции по $\ell$, что
% \[F_1\ldots F_{m-1}\vdash F_m\to B_{\ell}\,.(3)\]
% 1) $\ell=1$. Покажем, что $F_1\ldots F_{m-1}\vdash F_m\to B_{1}$. Для формулы $B_1$ как первого члена последовательности (1), являющейся выводом G из $F_1\ldots F_{m-1},F_m$, могут представиться следующие возможности:\\
% а) $B_1$ есть либо одна из аксиом, либо одна из гипотез $F_1,\ldots,F_{m-1}$. В этом случае вывод формулы $F_m\to B_1$ из гипотез $F_1,\ldots,F_{m-1}$ строится так:
% \[B_1,\quad B_1\to(F_m\to B_1),\quad F_m\to B_1\,,(4)\]
% ,где вторая формула есть аксиома (А1), а третья получена из первых двух по правилу modus ponens. Таким образом, в последовательность (2) перед первой формулой нужно добавить первую и вторую формулы из последовательности (4);\\
% б) $B_1$ есть гипотеза $F_m$. Тогда формула $F_m\to B_1$ принимает вид $F_m\to F_m$. Но согласно примеру 15.2, эта формула выводима не только из гипотез $F_1\ldots F_{m-1}$, но и просто из аксиом. В этом случае ее вывод, приведенный в примере 15.2, нужно вписать в последовательность (2) перед первой формулой;\\
% 2) $\ell \leqslant k$. Предположим, что утверждение (3) верно для всех $\ell \leqslant k$ и все необходимые выводы добавлены перед всеми k первыми формулами последовательности (2);\\
% 3) $\ell=k+1$. Покажем теперь, что утверждение (3) верно для $\ell=k+1$, т.е. справедлива выводимость: $F_1 \ldots F_{m-1}\vdash F_m\to B_{k+1}$. Для формулы $B_{k+1}$ как члена последовательности (1), являющейся выводом G из гипотез $F_1\ldots F_{m-1},F_m$, могут представиться следующие возможности: \\
% а), б) $B_{k+1}$ есть либо одна из аксиом, либо одна из гипотез $F_1 \ldots F_{m-1},F_m$. Данные возможности абсолютно аналогичны соответствующим возможностям из случая $\ell=1$ (там лишь нужно $ B_1$ заменить на $B_{k+1}$);\\
% в) $B_{k+1}$ получена из двух предыдущих формул $B_i,\,B_j$ последовательности (1) по правилу MP. Следовательно, $1 \leqslant i < k+1$, $1 \leqslant j < k+1$, а формула $B_j$ имеет вид: $B_i\to B_{k+1}$, то есть $B_j\equiv B_i\to B_{k+1}$. Поскольку $1 \leqslant i < k+1$ и $1 \leqslant j < k+1$, то формулы $F_m\to B_i$ и $F_m\to B_j$ стоят в последовательности (2) перед формулой $F_m\to B_{k+1}$ а следовательно, по предположению индукции, справедливы утверждения о том, что имеются вывод ы этих формул из гипотез $F_1,\ldots,F_{m-1}$. Выпишем эти выводы последовательно один за другим, они завершаются формулами $F_m\to B_i$ и $F_m\to (B_i\to B_{k+1})$ соответственно (напоминаем, что $B_j\equiv B_i\to B_{k+1}$):\\
% \begin{align*}
% 	& \cdots \cdots \cdots \cdots \cdots \cdots \cdots \cdots \cdots \cdots \cdots \\ 
% 	&(\alpha)\qquad\qquad F_m\to B_i\\ 
% 	& \cdots \cdots \cdots \cdots \cdots \cdots \cdots \cdots \cdots \cdots \cdots \\ 
% 	&(\alpha+\beta)\qquad\, F_m\to (B_i\to B_{k+1}). 
% \end{align*}
% Продолжим выводы следующими формулами:
% \begin{align*} 
% 	&(\alpha+\beta+1)\quad \bigl(F_m\to (B_i\to B_{k+1})\bigr) \to \bigl((F_m\to B_i)\to (F_m\to B_{k+1})\bigr);\\ 
% 	&(\alpha+\beta+2)\quad (F_m\to B_i)\to (F_m\to B_{k+1});\\ 
% 	&(\alpha+\beta+3)\quad F_m\to B_{k+1}. 
% \end{align*}
% \\
% Первая из формул есть аксиома (А2), вторая формула получена из первой и формулы ($\alpha+\beta$) по правилу MP, третья получена из второй и формулы (а) по правилу MP. Таким образом, построенная последовательность есть в этом случае вывод формулы $F_m\to B_{k+1}$ из гипотез $F_1\ldots F_{m-1}$. \\
% Итак, утверждение (3) верно для любого $\ell=1,2,\ldots,s$. При $\ell=s$ получаем (напоминаем, что $B_s\equiv G$): $F_1,\ldots,F_{m-1}\vdash F_m\to G$, что и требовалось доказать.

\end{document}