\documentclass[../main.tex]{subfiles}
\usepackage[utf8]{inputenc}
\usepackage[russian]{babel}
\usepackage{setspace, amsmath}
\usepackage{array}
\usepackage{amssymb}
\graphicspath{{\subfix{../images/}}}

\newtheorem{defin}{Определение:}


\begin{document}

% 28
\section{Формализованное исчисление предикатов (ФИП)}
\textbf{Определение:
} Язык ФИП (алфавит исчисления предикатов): предметных переменных $x_1,x_2\ldots$, предметных констант (символы выделенных элементов) $c_1,c_2 \ldots$, предикатных букв$ P'_1,P'_2 \ldots P'_k \ldots$, функциональных букв $f''_1,f''_2 \ldots f''_{\ell} \ldots$, а также знаков логических связок $\lnot и \land$, кванторов $\forall$, $\exists$ и скобок (,). При этом верхние индексы предикатных и функциональных букв указывают число аргументов соответственно предиката или функции, которые могут быть подставлены вместо этих букв. \\
\textbf{
	Определение:
} Формулы ФИП (как в логике предикатов): Сначала определяются термы. Ими являются отдельно взятые предметные переменные и константы, а также выражения вида $f^n(t_1,\ldots,t_n)$, где $f^n$ — n-местный функциональный символ; $t_1 \ldots t_n$ — термы\\
Справедливо:
\begin{itemize}
	\item если $P_n$ — предикатная буква, $t_1 \ldots t_n$ — термы, то $P_n(t_1 \ldots t_n)$ — формула; при этом все вхождения переменных в эту формулу называются свободными
	\item если $F_1,\,F_2$ — формулы, то формулами являются $\lnot F_1,\,(F_1\to F_2)$; причем все вхождения переменных, свободные в $F_1,\,F_2$, являются свободными и в формулах указанных видов; кроме того, можно считать, что в $F_1$ и $F_2$ нет предметных переменных, которые связаны в одной формуле и свободны в другой
	\item если $F_x$ — формула, содержащая свободные вхождения переменной x, то $(\forall x)(F(x))$ и $(\exists x)(F(x))$ — формулы; при этом вхождения переменной x в них называются связанными; вхождения же всех остальных предметных переменных в эти формулы остаются свободными (связанными), если они были свободными (связанными) в формуле $F(x)$ (формула $F(x)$ называется областью действия квантора)
	\item никаких других формул нет 
\end{itemize}
\textbf{Определение:
} Аксиомы ФИП:
\begin{itemize}
	\item аксиомы формализованного исчисления высказываний:
		\begin{align*}
			&\mathsf{(A1)}\colon\quad F\to (G\to F)\\ 
			&\mathsf{(A2)}\colon\quad \bigl(F\to (G\to F)\bigr)\to \bigl((F\to G)\to (F\to H)\bigr)\\ 
			&\mathsf{(A3)}\colon\quad (\lnot G\to\lnot F)\to \bigl((\lnot G\to F)\to G\bigr)
		\end{align*}
		$F,\,G,\,H$ - любые формулы исчисления предикатов.
	\item предикатные аксиомы (схемы аксиом), т.е. аксиомы с кванторами - аксиомы Бернайса:
		\begin{align*}
			&\mathsf{(PA1)}\colon\quad (\forall x)(F(x)\to F(y)) \\ 
			&\mathsf{(PA2)}\colon\quad F(y)\to (\exists x)(F(x))
		\end{align*}
		F(x) — любая формула, содержащая свободные вхождения x, причем ни одно из них не находится в области действия квантора по y (если таковой имеется)
\end{itemize}
\textbf{Определение:
} Правила вывода ФИП:
\begin{align*}
	&\text{правило отсечения}(\textit{modus ponens}):\frac{F, F \to G}{G} \\ 
	&\forall-\text{правило (обобщения)}: \frac{F\to G(x)}{F\to (\forall x)(G(x))} \\
	&\exists-\text{правило (конкретизации)}: \frac{G(x)\to F}{(\exists x)(G(x))\to F}
\end{align*}

% 29
\section{Свойства ФИП}
\textbf{Факт:
} Полнота, непротиворечивость, разрешимость.\\
\textbf{Теорема:
} Формализованное исчисление предикатов непротиворечиво: одновременно не выводимы две противовоположные формулы ФИП. \\
\textbf{Теорема:
} О разрешимости: формализованное исчисление предикатов неразрешимо \\
\textbf{Теорема:
} о полноте: в ФИП всякая теорема выводима.


\end{document}