\documentclass[../main.tex]{subfiles}
\usepackage[utf8]{inputenc}
\usepackage[russian]{babel}
\usepackage{setspace, amsmath}
\usepackage{array}
\usepackage{amssymb}
\graphicspath{{\subfix{../images/}}}

\newtheorem{defin}{Определение:}

\begin{document}

% 30
\section{Понятие формальной аксиоматической теории}
\textbf{Определение:
} Формальная аксиоматическая теория ФАТ = $\{A,C,P\}$
\begin{itemize}
	\item задан алфавит теории A - А
	\item задан алгоритм, который для каждой формулы может проверить корректность записи, синтаксис - С
	\item задан алгоритм доказательства всех теорем, семантическое правило - P
\end{itemize}
Истина - все те теоремы что не противоречат исходным теориям.
Формализованное исчисление высказываний может служить примером формальной аксиоматической теории. \\
\textbf{Определение:
} Вывод в формальной аксиоматической теории - это всякая последовательность $B_1 \ldots B_n$ формул этой теории, такая, что $B_i$ есть либо аксиома теории, либо непосредственное следствие каких-либо предыдущих формул по одному из правил вывода. Формула F теории называется теоремой, если существует вывод в T, последней формулой которого является F.

% % 30
% \section{Понятие формальной аксиоматической теории}
% \textbf{Определение:
% } Формальная аксиоматическая теория Г считается определенной, если выполнены следующие условия:
% \begin{itemize}
% 	\item задан алфавит теории A, представляющий собой некоторое счетное множество символов. Конечные последовательности символов алфавита теории T называются словами или выражениями теории T;
% 	\item имеется подмножество выражений теории T, называемых формулами теории T (обычно имеется эффективная процедура, позволяющая по данному выражению определить, является ли оно формулой);
% 	\item выделено некоторое множество формул, называемых аксиомами теории T (обычно имеется эффективная процедура, позволяющая по данной формуле определить, является ли она аксиомой);
% 	\item имеется конечное множество $R_1 \ldots R_n$ отношений между формулами, называемых правилами вывода. При этом для каждого $R_i(1 \leqslant I \leqslant n)$ существует целое положительное j, такое, что для каждого множества, состоящего из j формул, и для каждой формулы F эффективно решается вопрос о том, находятся ли данные j формул в отношении $R_i$ с формулой F, и если да, то F называется непосредственным следствием данных j формул по правилу $R_i$.
% \end{itemize}
% Формализованное исчисление высказываний может служить примером формальной аксиоматической теории. \\
% \textbf{Определение:
% } Вывод в формальной аксиоматической теории T - это всякая последовательность $B_1 \ldots B_n$ формул этой теории, такая, что для любого $i(1 \leqslant i \leqslant n)$ формула $B_i$ есть либо аксиома теории T, либо непосредственное следствие каких-либо предыдущих формул по одному из правил вывода. Формула F теории T называется теоремой этой теории, если существует вывод в T, последней формулой которого является F;

% 31
\section{Формальная арифметика (ФА) Пеано
}
\textbf{Вторая трактовка:
} Система аксиом Пеано формальной арифметики состоит из 7 индивидуальных аксиом и одной бесконечной серии аксиом (эту серию аксиом называют схемой индукции):
\begin{itemize}
	\item $\forall x(\lnot(0 = S_x))$
	\item $\forall x \forall (S_x = S_y \to x = y)$
	\item $\forall x(\lnot (x = 0) \to \exists y(x = S_y))$
	\item $\forall x(x + 0 = x)$
	\item $\forall x \forall y(x + S_y = S(x + y))$
	\item $\forall x(x \cdot  0 = 0)$
	\item $\forall x \forall y(x \cdot (S_y) = (x \cdot y) + x)$
	\item $(\phi (0) \land \forall x(\phi (x) \to \phi (S_x))) \to \forall x \phi (x)$,
\end{itemize}
где $\phi (x)$ произвольная формула со свободной переменной x.

% 32
\section{Математическая индукция}
\textbf{Определение:
} Метод математической индукции — специальный метод доказательства, применяемый для доказательства истинности утверждений типа $(\forall x\in \mathbb{N})(P(x))$, то есть $(\forall x)(x\in \mathbb{N}\to P(x))$. Такие утверждения выражают тот факт, что некоторое свойство P присуще каждому натуральному числу. \\
\textbf{Аксиома:
} Если свойством P обладает число 1 и для всякого натурального числа из того, что оно обладает этим свойством, следует, что и непосредственно следующее за ним натуральное число также обладает им, то и всякое натуральное число обладает свойством Р:
\[\Bigl(P(1)\land (\forall x)\bigl(P(x)\to P(x+1)\bigr)\Bigr)\to (\forall y)\bigl(P(y)\bigr)\]
\textbf{Факт:
} Cхема доказательства методом математической индукции может быть представлена следующим образом:
\begin{itemize}
	\item (1): P(1) — устанавливается проверкой
	\item (2): $(\forall x)\bigl(P(x)\to P(x+1)\bigr)$ — доказывается
	\item (3): $P(1)\land (\forall x)\bigl(P(x)\to P(x+1)\bigr)$ - из (1), (2) по правилу введения конъюнкции;
	\item (4): $\bigl(P(1)\land (\forall x)\bigl(P(x)\to P(x+1)\bigr)\bigr)\to (\forall y)\bigl(P(y)\bigr)$ — аксиома индукции
	\item (5): $(\forall y)\bigl(P(y)\bigr)$ — из (3), (4) по правилу modus ponens.
\end{itemize}
P(1) - базой индукции, предположение об истинности утверждения P(x) — предположение индукции, доказательство истинности утверждения P(x+1) — шаг индукции.

% 33
\section{Элиминация кванторов}
\textbf{Определение:
} Элиминация кванторов: для формулы существовует такая бескванторная формула B, что в данной теории доказуема (выводима) эквивалентность $A\leftrightarrow B$. \\
Алгоритм:
\begin{itemize}
	\item привести в предварённую нормальную форму
	\item заменить все переменные предикатов всеобщности $\forall$ на $x_k$ переменные множества
	\item заменить все переменные предикатов существования $\exists$ на $n$-местные функциональные символы $f_{k}^n$
\end{itemize}
\textbf{Пример:
}\begin{enumerate}
	\item $\forall a \forall b \forall f \exists C[ (O(a)   \land  O(b)   \land  N(f,a,b)   \land  D(f,a,b)) \to (P(C,a,b)   \land  G(f,a,b,C))]$
	\item $\forall b \forall f \exists C[ (O(x_1 )   \land  O(b)   \land  N(f,x_1,b)   \land  D(f,x_1,b)) \to (P(C,x_1,b)   \land  G(f,x_1,b,C))]$
	\item $\forall f \exists C[ (O(x_1 )   \land  O(x_2 )   \land  N(f,x_1,x_2 )   \land  D(f,x_1,x_2 )) \to (P(C,x_1,x_2 )   \land  G(f,x_1,x_2,C))]$
	\item  $\exists C[ (O(x_1 )   \land  O(x_2 )   \land  N(x_3,x_1,x_2 )   \land  D(x_3,x_1,x_2 )) \to (P(C,x_1,x_2 )   \land  G(x_3,x_1,x_2,$C))]
	\item $(O(x_1 )   \land  O(x_2 )   \land  N(x_3,x_1,x_2 )   \land D(x_3,x_1,x_2 )) \to (P(f_{1}^3,x_1,x_2 )   \land  G(x_3,x_1,x_2,f_{1}^3 ))$
	
\end{enumerate}

% 34
\section{Понятие гёделевской нумерации}
\textbf{Определение:
} Пусть K — теория первого порядка, содержащая переменные $x_1,x_2 \ldots$, предметные константы $a_{1},a_{2} \ldots$ функциональные символы $f_{k}^{n}$ и предикатные символы $A_{k}^{n}$, где k — номер, а n — арность функционального или предикатного символа. \\
Способ обозначения объектов ФА, например g(u):
\begin{itemize}
	\item $G(() = 3$
	\item $G()) = 5$
	\item $G(,) = 7$
	\item $G(\lnot) = 9$
	\item $G(\to) = 11$
	\item $G(x_{k})=5+8k,\ k=1,2,\ldots $
	\item $G(a_{k})=7+8k,\ k=1,2,\ldots $
	\item $G(f_{k}^{n})=9+8\cdot 2^{n}3^{k},\ k,n\geqslant 1;$
	\item $G(A_{k}^{n})=11+8\cdot 2^{n}3^{k},\ k,n\geqslant 1.$
\end{itemize}
Гёделев номер произвольной последовательности $e_{0},\ldots ,e_{r}$ выражений определим следующим образом: 
\[G(e_{0}\ldots e_{r})=2^{{G(e_{0})}}\cdot 3^{{G(e_{1})}}\cdot \ldots \cdot p_{r}^{{G(e_{r})}}\]
\textbf{Пример:
}
\[G(A_{1}^{2}(x_{1},x_{2}))=2^{{G(A_{1}^{2})}}\cdot 3^{{G(()}}\cdot 5^{{G(x_{1})}}\cdot 7^{{G(,)}}\cdot 11^{{G(x_{2})}}\cdot 13^{{G())}}=2^{{107}}\cdot 3^{{3}}\cdot 5^{{13}}\cdot 7^{{7}}\cdot 11^{{21}}\cdot 13^{{5}}\]

% 35
\section{Первая теорема Гёделя о неполноте ФА}
\textbf{Теорема:
} Гёделя о неполноте (первая): \[\exists \Phi: \ \vDash \Phi \Rightarrow \ \nvdash \Phi \]
Eсли ФА непротиворечива, то она не полна. \\
\textbf{Пример:
} 
\begin{align*}
	&\Phi \doteqdot \textit{``не существует доказательства формулы $\Phi$''} \\ 
	&\Phi \doteqdot \bigl( \lnot \exists G\left[ Prf(\Phi)\right] = n \bigr)
\end{align*}

% \textbf{Определение:
% } Гёделя о неполноте: Всякая $\omega$-непротиворечивая и адекватная формальная арифметика не является полной. \\
% \textbf{Определение:
% } Формальная арифметика называется $\omega$ -непротиворечивой, если в ней нет такой формулы G(x) с единственной свободной предметной переменной x, что для всех натуральных чисел n справедливы теоремы $\vdash G(n^{\ast})$ и $\vdash\lnot (\forall x)(G(x))$. \\
% \textbf{Определение:
% } Формальная арифметика - адекватна, если для каждого перечислимого множества $Q$ натуральных чисел существует вполне представимый в этой арифметике предикат $P(x,y)$ такой, что $Q=\bigl\{x\colon (\exists y)(\lambda [P(x,y)]=1)\bigr\}$. \\
% \textbf{Определение:
% } Под полнотой формальной арифметики будем понимать абсолютную полноту, т.е. если для каждой замкнутой формулы F этой теории либо она сама, либо ее отрицание является теоремой этой теории: $\vdash F$ или $\vdash\lnot F$.

% 36
\section{Вторая теорема Гёделя о неполноте ФА }
\textbf{Теорема:
} Гёделя о неполноте (вторая): \[\exists \varPhi: \ \nvdash \varPhi \ \Leftrightarrow \ \vDash \varPhi\]
Eсли ткорема не выводима, то она истина, и наоборот. \\
% \textbf{Пример:
% } 
% \begin{align*}
% 	&\Phi \doteqdot \textit{``не существует доказательства формулы $\Phi$''} \\ 
% 	&\Phi \doteqdot \bigl( \lnot \exists G\left[ Prf(\Phi)\right] = n \bigr)
% \end{align*}


% 37
\section{Теорема Чёрча о неразрешимости ФА}
\textbf{Теорема
} Не существует общего алгоритма, позволяющего за конечное число шагов определить, является ли заданная формула формальной арифметики доказуемой, или нет.

% 38
\section{Теорема Тарского о понятии истинности в ФА}
\textbf{Теорема:
} Не всё что истинно является доказуемым. \\
\textbf{Пример:
} Точка - объект, не имещий никаких свойств в математике.


\end{document}