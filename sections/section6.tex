\documentclass[../main.tex]{subfiles}
\usepackage[utf8]{inputenc}
\usepackage[russian]{babel}
\usepackage{setspace, amsmath}
\usepackage{array}
\usepackage{amssymb}
\graphicspath{{\subfix{../images/}}}

\newtheorem{defin}{Определение:}


\begin{document}

% 39
\section{Машина Тьюринга (МТ)}
\textbf{Определение:
} Машина Тьюринга - не физическая машина, а математический объект, как функция, производная..., моделирующая реальные процессы. Работатет с лентой ячеек. \\
Обладает:
\begin{itemize}
	\item внешним алфавитом $A=\{a_0,a_1 \ldots a_n\}$ - все доступные символы. $a_0$ - пустой символ
	\item множеством состояний $Q=\{q_0,q_1 \ldots q_m\}$, $q_1$ - начало работы, $q_0$ - остановка.
	\item программой(функциональной схемой), которая определяет работу машины и состоит из команд вида $K(i,j) \colon q_i a_j\to q_k a_lX$, где $X\in\{\text{С,П,Л}\}$ 
\end{itemize}
Суть команды: имея состояние $q_i$, машина стирает значение $a_j$ обозреваемой ячейки, записывает символ $a_l$, переходит в новое состояние $q_k$ и перемещается в другую ячейку (или нынешнюю), согласно одному из указаний $\{\text{С - стоять,П - правая,Л - левая}\}$ \\
Только одна команда соответствует $q_ia_j$, следовательно программа располагает $m(n+1)$ командами.

% 40
\section{Представление функциональной схемы МТ в табличном виде}
\textbf{Факт:
} Имея алфавит $A= \{a_0,a_1,a_2\}$, внутренние состояния $Q=\{q_0,q_1,q_2,q_3\}$ и программу 
$q_1a_1\to q_1a_1\text{Л},~ q_2a_1\to q_3a_1\text{П},~ q_3a_1\to q_1a_1\text{Л},~ q_1a_1\to q_2a_1\text{Л},~ q_2a_1\to q_2a_1\text{Л}, q_3a_1\to q_3a_1\text{П},~ q_1a_0\to q_0a_1,~ q_2a_0\to q_2a_0 \text{Л},~ q_3a_0\to q_3a_0\text{П}$, 
для работы с машиной удобно составить таблицу:
\begin{center}
	\begin{tabular}{|c|c|c|c|}
		\hline 
		$A | Q$ & $q_1$& $q_2$& $q_3$\\
		\hline 
		$a_0$& $q_0a_1$& $q_2a_0 \text{Л}$& $q_3a_0\text{П}$\\
		\hline 
		$a_1$& $q_1a_1\text{Л}$& $q_1a_2\text{П}$& $q_1a_1\text{Л}$\\
		\hline  
		$a_2$& $q_2a_1\text{Л}$& $q_2a_2\text{Л}$& $q_3a_2\text{П}$\\ 
		\hline 
	\end{tabular}
\end{center} 
Команды записываются в ячейки на перечении $q_ia_j$, так как пара $q_ia_j$ однозначно их определяет.

% 41
\section{Понятие функции вычислимой по Тьюрингу}
\textbf{Определение:
} Функция вычислима по Тьюрингу, если существует машина Тьюринга, которая вычисляет ее значения, если функция определена для данных аргументов, которая работает вечно, если функция не определена.
\textbf{Пример:
} $f(x)=\frac{x}{10}$ - определена только для чисел кратных 10ти: 
\begin{equation*}
    \text{МТ} (x) = 
    \begin{cases}
		\text{оставить на ленте} \frac{x}{10} \text{, если x кратно десяти}\\
        \text{работать бесконечно}\text{, если x не кратно десяти}
    \end{cases}
\end{equation*}

% 42
\section{Простейшие функции вычислимые по Тьюрингу}
\textbf{Примеры:
} \begin{itemize}
	\item $I_m^n(x_1,x_2,\ldots,x_n)=x_m$ - функции-проекторы
	\item $S(x) = x + 1$ - добавление единицы
	\item $O(x) = 0$ нуль-функция 
	\item $K(x, y) = x+y$ функция-сложения
	\item $J(x, y) = x*y$ функция-умножения
\end{itemize} 
Имея только эти функции, при помощи композиции, уже можно получить множество других функций.

% 43
\section{Тезис Тьюринга}
\textbf{Определение:
} Тезис(основная гипотеза теории алгоритмов) Тьюринга: для нахождения значений функции, заданной в некотором алфавите, тогда и только тогда существует какой-нибудь алгоритм, когда функция является вычислимой по Тьюрингу.

% 44
\section{Композиция нескольких МТ}
\textbf{Определение:
} Композиция (произведение) машины $\Theta_1$ на машину $\Theta_2$ - новая машина $\Theta_3$ с тем же внешним алфавитом $\{a_0,a_1 \ldots a_m\}$, состояниями $\{q_0,q_1 \ldots q_n,q_{n+1},\ldots,q_{n+t}\}$ и программой, которая составлена заменой во всех командах первой машины состояния $q_0$ на состояние $q_{n+1}$ второй машины. \\
Таким образом, композицию можно получить просто продолжив работу одной машины работой другой.

\end{document}