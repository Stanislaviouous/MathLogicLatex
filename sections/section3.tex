\documentclass[../main.tex]{subfiles}
\usepackage[utf8]{inputenc}
\usepackage[russian]{babel}
\usepackage{setspace, amsmath}
\usepackage{array}
\usepackage{amssymb}
\graphicspath{{\subfix{../images/}}}

\newtheorem{defin}{Определение:}


\begin{document}

% 20
\section{Понятие предиката}
\textbf{Определение:
} Определенный на множествах $M_1,M_2 \ldots M_n$ n-местным предикат - это предложение, содержащее $n$ переменных $x_1,x_2 \ldots x_n$, превращающееся в высказывание при подстановке вместо этих переменных любых конкретных элементов из множеств $M_1,M_2 \ldots M_n$ соответственно. \\
\textbf{Пример:
} ``Действительное число x кратно 2378''

% 21
\section{Множество истинности предиката}
\textbf{Определение:
} Множество истинности предиката P заданного на множествах $M_1,M_2 \ldots M_n$ - это совокупность всех значений таких, что данный предикат $P(a_1,a_2,\ldots,a_n)$ обращается в истинное высказывание. Это множество $M^{+}$:
\[M^{+}= \bigl\{(a_1,a_2,\ldots,a_n)\colon\, \lambda \bigl(P(a_1,a_2, \ldots, a_n)\bigr)= 1\bigr\}\]

% 22
\section{Классификация предикатов}
\textbf{Определение:
} Предикат P, заданный на множествах $M_1,M_2 \ldots M_n$:
\begin{itemize}
	\item Тождественная истинна (ложь), если при любой подстановке вместо переменных $x_1,x_2,\ldots,x_n$ любых конкретных предметов $a_1,a_2,\ldots,a_n$ из множеств $M_1,M_2,\ldots,M_n$ его логическое значение 1 (0)
	\item Выполнимый (опровержимый), если существует хотя бы один набор конкретных предметов $a_1,a_2 \ldots a_n$ из множеств $M_1,M_2 \ldots M_n$, при подстановке которых вместо соответствующих предметных переменных предикат P принимает логическое значение 1 (0).
	\item Общезначимый - тождественная истинна на всякой области, на любой модели.
\end{itemize}

% 23
\section{Понятие формулы логики предикатов (ЛП)}
\textbf{Определение:
} Алфавит формулы: предметные переменные; нульместные предикатные переменные: $P,Q,R,P_i,Q_i,R$; n-местные $(n\geqslant1)$ предикатные переменные: $R(,\ldots,),P_i(,\ldots,)$; символы логических операций: $\lnot,~ \land,~ \lor,~ \to,~ \leftrightarrow$; кванторы: $\forall, \exists$; вспомогательные символы: (,). \\
\textbf{
	Определение:
} Формула логики предикатов - сложное предложение определённое на множестве, составленное из элементов алфавита:\\
Справедливо:
\begin{itemize}
	\item Каждая нульместная предикатная переменная есть формула.
	\item Если $P(~,\ldots,)$ — n-местная предикатная переменная, то $P(x_1,\ldots,x_n)$ есть формула, в которой все предметные переменные $x_1,\ldots,x_n$ свободны.
	\item Если $F_1,\,F_2$ — формулы и если предметные переменные, входящие одновременно в обе эти формулы, свободны в каждой из них, то выражения:$\lnot F$,$(F_1\land F_2),\quad (F_1\lor F_2),\quad (F_1\to F_2),\quad (F_1\leftrightarrow F_2)$ - тоже формулы. 
	\item Если F — формула и x — предметная переменная, входящая в F свободно, то выражения $(\forall x)(F) и (\exists x)(F)$ также являются формулами, в которых переменная x связанная, а все остальные предметные переменные, входящие в формулу F свободно или связанно, остаются и в новых формулах соответственно такими же.
	\item никаких других формул логики предикатов нет.
\end{itemize}
\textbf{Пример:
}\[(\forall a \lnot P(a) \land \exists x(P (x) \land Q(x, y)))\]

% 24
\section{Свободные и связанные переменные формулы ЛП}
\textbf{Определение:
} Свободные переменные формулы ЛП - не входит в область действия квантора по этой переменной. - не применена кванитификация. \\
\textbf{Определение:
} Связанная переменные формулы ЛП - это переменные, над которыми применена кванитификация. \\
\textbf{Пример:
} \[\forall  x \bigl[(\exists y)\bigl(P(x, y)\bigr) \to Q(x, y, z)\bigr]\]
x - связанная \\
y - частично связанная \\
z - свободная \\

% 25
\section{Теорема о приведенной форме для формулы ЛП}
\textbf{Определение:
} Приведенной формой для формулы логики предикатов называется такая равносильная ей формула, в которой из операций алгебры высказываний имеются только операции $\lnot,\,\land,\,\lor$, причем знаки отрицания относятся лишь к предикатным переменным и к высказываниям.\\
\textbf{Теорема:
} Для каждой формулы логики предикатов существует приведенная форма. \\
\textbf{Доказательство:
} Проводтся методом математической индукции по числу логических связок в формуле (включая кванторы общности и существования).

% 26
\section{Теорема о предварённой нормальной форме для формулы ЛП}
\textbf{Определение:
} Предваренной нормальной формой для формулы логики предикатов называется такая ее приведенная форма, в которой все кванторы стоят в начале, а область действия каждого из них распространяется до конца формулы, т. е. это формула вида $(K_1x_1)\ldots(K_mx_m)\bigl(F(x_1,\ldots,x_n)\bigr)$, где $K_i$ есть один из кванторов $\forall$ или $\exists (i=1,\ldots,m),~m \leqslant n$, причем формула F не содержит кванторов и является приведенной формулой. (Заметим, что кванторы в формуле могут отсутствовать вовсе.) \\
\textbf{Теорема:
} Для каждой формулы логики предикатов существует предваренная нормальная форма. \\
\textbf{Доказательство:
} Проводтся по индукции, следуя правилу построения формул логики предикатов. \\

% 27
\section{Кванторные законы логики предикатов}
1. Законы де Моргана для кванторов: \\
1.1 $\lnot (\forall x)(P(x)) \leftrightarrow (\exists x)(\lnot P(x))$ \\
1.2 $\lnot (\exists x)(P(x)) \leftrightarrow (\forall x)(\lnot P(x))$ \\
2. Выражение кванторов одного через другой \\
2.1 $(\forall x)(P(x)) \leftrightarrow \lnot (\exists x)(\lnot P(x))$ \\
2.2 $(\exists x)(P(x)) \leftrightarrow \lnot (\forall x)(\lnot P(x))$ \\
3. Пронесение кванторов через конъюнкцию и дизъюнкцию \\
3.1 $(\forall x) (P(x)\land Q(x)) \leftrightarrow (\forall x)(P(x)) \land (\forall x)(Q(x))$ \\
3.2 $(\exists x) (P(x)\lor Q(x)) \leftrightarrow (\exists x) (P(x)) \lor (\exists x)(Q(x))$ \\
3.3 $(\forall x)(P(x)\lor Q) \leftrightarrow (\forall x)(P(x))\lor Q$ \\
3.4 $(\exists x)(P(x)\land Q) \leftrightarrow (\exists x)(P(x))\land Q$ \\
4. Пронесение кванторов через импликацию предикатов \\ 
4.1 $(\forall x)(P(x)\to Q) \leftrightarrow ((\exists x)(P(x))\to Q)$ \\
4.2 $(\exists x)(P(x)\to Q) \leftrightarrow ((\forall x)(P(x))\to Q)$ \\
4.3 $(\forall x)(Q\to P(x)) \leftrightarrow (Q\to (\forall x)(P(x)))$ \\
4.4 $(\exists x)(Q\to P(x)) \leftrightarrow (Q\to (\exists x)(P(x)))$ \\
5. Законы коммутативности для кванторов \\ 
5.1 $(\forall x)(\forall y)(P(x,y)) \leftrightarrow (\forall y)(\forall x)(P(x,y))$ \\
5.2 $(\exists x)(\exists y)(P(x,y)) \leftrightarrow (\exists y)(\exists x)(P(x,y))$ \\
5.3 $(\exists y)(\forall x)(P(x,y)) \to (\forall x)(\exists y)(P(x,y))$ \\
6. Законы универсальной конкретизации и экзистенциального обобщения \\ 
6.1 $(\forall x)(F(x))\vDash F(y)$ \\
6.2 $F(y)\vDash (\exists x)(F(x))$ \\

\end{document}