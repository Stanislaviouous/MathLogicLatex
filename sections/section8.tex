\documentclass[../main.tex]{subfiles}
\usepackage[utf8]{inputenc}
\usepackage[russian]{babel}
\usepackage{setspace, amsmath}
\usepackage{array}
\usepackage{amssymb}
\graphicspath{{\subfix{../images/}}}

\newtheorem{defin}{Определение:}


\begin{document}
%
\section{Алгоритм как словарная функция}
Так как любая задача и результат её выполнения могут быть представлены в виде слова в некотором алфавите, то алгоритм можно определить как словарная функция:
\[ A:B \to C\]
Алгоритм как функция отображает входные данные из алфавита $B$ в алфавит $C$.

%
\section{Свойства алгоритма}
\textbf{Факт:
} \begin{enumerate}
	\item Наличие входных и выходных данных. Данные должны быть допустимыми, алгоритм может выдать ответ за конечное время и количество операций, зациклиться или закончится ошибкой - всё из-за входных
	\item Дискретность – есть чёткая последовательность операций, выполняя которую, мы получаем конкретный результат
	\item Детерминированность - один и тот же ответ для всех пользователей(результат работы алгоритма зависит только от входных данных)
	\item Массовый характер – может быть применен к широкому множеству больших данных (позволяет решить целый класс однотипных задач)
\end{enumerate}

%
\section{Теорема Райса}
\textbf{Теорема:
} Если заданн класс С функции, нельзя понять, что делает программа не анализируя её.  Пусть C — любой непустой класс вычислимых функций, тогда не существует алгоритма, который бы по номеру x функции $f_x$ определял бы, принадлежит $f_x$ классу C или нет. Иначе говоря, множество $\{x\colon\, f_x\in C\}$ неразрешимо.

%
\section{Алгоритмически неразрешимые массовые проблемы}
\textbf{Определение:
} Алгоритмически неразрешимая задача - задача, для которой не существует алгоритма, который позволяет единым методом решить любую задачу из класса. Неразрешимость какой-либо массовой задачи как правило доказывается путем ее сведения к другой задаче, про которую уже известно, что она является неразрешимой. \\
\textbf{Пример:
} Проблемы определения самоприменимости и проблема остановки.
\begin{itemize}
	\item \textbf{Теорема:
	} О неразрешимости проблемы остановки:  Не существует алгоритма, который для заданного алгоритма $A$ и слова $w$ проверяет, что значение $A(w)$ определено - остановится ли машина или нет.
	\item \textbf{Теорема:	
	} О неразрешимости определения самоприменимости алгоритмов - не существует алгоритма, который по заданному определяет, является ли алгоритм A самоприменимым или нет. Самоприменимый алгоритм, например, для машины Тьюринга - это возможностью обрабатывать свой собственный код. $ T(\langle T \rangle) = V $
\end{itemize}
\textbf{Другие примеры:
} \begin{itemize}
	\item Проблема соответствий Поста
	\item Выполнимость оператора
	\item Проблема равенства слов
\end{itemize}

%
\section{Временнáя функция сложности алгоритма}
\textbf{Определение:
} Временнáя функция сложности алгоритма - такая функция $C(r)$ от объёма входных данных $r$, что показывает количество времени, которое
необходимо для завершения работы алгоритма на входных данных. В данном случае, время абстрактно (не секунды, минуты - шаг).

%
\section{Сравнение функций сложности на основе их асимптотических оценок}
\textbf{Факт:
} Если $C(r)$ - временнáя функция сложности от объёма входных данных r и 
\[\lim_{r \to \infty }{\frac{C_1(r)}{C_2(r)}} = X\]
\begin{itemize}
	\item $X = \infty \ \Rightarrow \ C_1(r) = \Omega \bigl(C_2(r)\bigr)$
	\item $X = \textit{const} \ \Rightarrow \ C_1(r) = O \bigl(C_2(r)\bigr)$
	\item $X = \textit{1} \ \Rightarrow \ C_1(r) \equiv C_2(r)$ (не равны, а асимптотически эквивалентны)
\end{itemize}

%
\section{Шкала асимптотической сложности алгоритмов}
\[\stackrel{\textit{ещё легче}}{\ln{\ln{r}} \ \cdots \ \ln{r}  \ \cdots \ }  < \stackrel{\textit{полином}}{ \ \cdots \ \frac{1}{r}  \ \cdots \ r^k  \ \cdots \ } < \stackrel{\textit{сложно}}{\ \cdots \  e^r  \ \cdots \ r!  \ \cdots \ r^r}\]

%
\section{Классификация массовых проблем по сложности}
\textbf{Факт:
} Массовые проблемы: алгоритмически разрешимые, алгоритмически неразрешимые. \\
\begin{center}
	\begin{tabular}{c c c}
		&I \textit{ класс}& \\
		&II \textit{ класс}&  P \\
		\hline
		exp(r)&III \textit{ класс}& \\ 
		&IV \textit{ класс}&  NP \\
	\end{tabular}
\end{center}
I $\textit{ класс} : C(r) \leq k*r$ \\
II $\textit{ класс} : C(r) \leq p_k(r)$ - полином \\
III $\textit{ класс}: C(r) \leq e^r$ \\
IV $\textit{ класс} : C(r) \geq e^r$  \\


\section{Понятие P-разрешимой массовой проблемы}
\textbf{Определение:
} Проблема P - это проблема решения с помощью полиномиального алгоритма, которая может быть решена за полиномиальное время. 

\section{Труднорешаемые массовые проблемы}
\textbf{Определение:
} NP(недетерминированный полином) - это класс труднорешаемых проблем, решение которых можно проверить за полиномиальное время. Так как задачу из этого класса можно представить как набор бесконечных детерминированных P-задач, то решить её за полиномиальное время можно, если отгадать результат и проверить. \\
Класс $P \subseteq NP$ – это множество задач, для которых существуют полиномиальные алгоритмы решения. \\
\textbf{Определение:
} NP-Complete - пока ни один алгоритм с полиномиальным временем не может ее решить, результат отрицательный (она не решена).

\end{document}