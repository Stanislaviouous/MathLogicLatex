\documentclass[../main.tex]{subfiles}
\usepackage[utf8]{inputenc}
\usepackage[russian]{babel}
\usepackage{setspace, amsmath}
\usepackage{array}
\usepackage{amssymb}
\graphicspath{{\subfix{../images/}}}

\newtheorem{defin}{Определение:}

\begin{document}

% 45
\section{Понятие функции вычислимой по Чёрчу}
Класс алгоритмически вычислимых частичных функций совпадает с классом всех частично рекурсивных функций. \\
\textbf{Факт:
} Теория рекурсивных функций включает следующие классы функций: класс примитивных рекурсивных функций, класс общерекурсивных функции и класс частично рекурсивных функций.
Теория рекурсивных функций:\\
	1) Задается базис элементарных функций:
 \begin{itemize}
	\item $S(x)=x+1$ функция следования
    \item $O(x)=0$ нуль-функция (константа)
    \item $I_m^n(x_1,x_2 \ldots x_n)=x_m$ функция-проектор
 \end{itemize}
	2) Определяются специальные операции над функциями:
 \begin{itemize}
	\item оператор суперпозиции 
	\[\varphi(x_1 \ldots x_n)= f \bigl(g_1(x_1 \ldots x_n),\,\ldots,\, g_m(x_1 \ldots x_n)\bigr)\]
    \item оператор примитивной рекурсии 
    \begin{align*}
		&\varphi(x_1 \ldots x_n,0)= f(x_1 \ldots x_n);\\ 
		&\varphi(x_1 \ldots x_n, y+1) = g \bigl(x_1 \ldots x_n, y, \varphi(x_1 \ldots x_n,y)\bigr). 
	\end{align*}
    \item оператор минимизации
    \begin{align*}
		&f_1(x_1 \ldots x_n, 0) \ne f_2(x_1 \ldots x_n, 0) \\ 
		& \cdots \\
		&f_1(x_1 \ldots x_n, y - 1) \ne f_2(x_1 \ldots x_n, y - 1) \\ 
		&f_1(x_1 \ldots x_n, y) = f_2(x_1 \ldots x_n, y) \\ 
		&\varphi(x_1 \ldots x_n) = \text{ряд} \bigl[ f_1(x_1 \ldots x_n, y) = f_2(x_1 \ldots x_n, y) \bigr]
	\end{align*}
 \end{itemize}
	3) В результате применения определенного количества операций к элементарным функциям, строятся другие

% 46
\section{Простейшие примитивно рекурсивные функции}
\textbf{Определение:
} Функция примитивно рекурсивна, если она может быть получена из исходных простейших функций (нуль, следование, проектор) с помощью конечного числа применений операторов суперпозиции и примитивной рекурсии.

% 47
\section{Понятие общерекурсивной функции}
\textbf{Определение:
} Общерекурсивные функции — это подмножество частично рекурсивных функций, определённых для всех значений аргументов. \\
\textbf{Факт:
} Задача определения того, является ли частично рекурсивная функция с данным описанием общерекурсивной или нет, алгоритмически неразрешима.

% 48
\section{Понятие частично рекурсивной функции}
\textbf{Определение:
} Функция частично рекурсивна, если она может быть получена из простейших функций $O,\,S,\,I_m^n$ с помощью конечного числа применений суперпозиции, примитивной рекурсии и оператора минимизации. \\
Как было показано Гёделем, частично рекурсивные функции совпадают с множеством вычислимых функций.

% 49
\section{Тезис Чёрча}
\textbf{Определение:
} Тезиса Чёрча: Числовая функция тогда и только тогда алгоритмически (или машинно) вычислима, когда она частично рекурсивна. \\
Класс алгоритмически вычислимых частичных функций совпадает с классом всех частично рекурсивных функций.

% 50
\section{Теорема об эквивалентности множества функций вычислимых по Тьюрингу и множества функций вычислимых по Чёрчу}
\textbf{Теорема:
} Функция вычислима по Тьюрингу тогда и только тогда, когда она вычислима по Чёрчу. \\
\textbf{Теорема: 
} Класс всех частично рекурсивных функций совпадает с классом всех функций, вычислимых по Тьюрингу. \\
\textbf{Теорема:
} Множества функций вычислимых по Тьюрингу и множества функций вычислимых по Чёрчу эквивалентны

\end{document}